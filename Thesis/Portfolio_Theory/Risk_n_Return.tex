We want to establish the classical Portfolio Theory as a starting point of our work. Understanding the advantages and drawbacks of this theory will allow us to better explain some choices that have been made while developing our portfolio model. As for any rational investor we try to think in terms of risk and return. The return is a measure of how much money we will make investing in our portfolio, while the risk is a measure of how unstable our money-making model is. The latter is traditionally measured in terms of variance of portfolio returns, that's why we refer to a \textit{Mean-Variance} problem. We will illustrate later how covariance between assets in a portfolio plays a crucial role. The idea is that given a certain level of risk we will try to get out the highest return, or on the other direction, given a certain level of desired return, we will try to minimize the risk of our portfolio. The traditional Markowitz theory establishes a clear and intuitive relationship between risk and return that allows any investor to find an optimal allocation given a certain level of risk or return.\\
For our specific case, where strategies are our assets to which we allocate risk we will consider as return the amount of PnL given by a strategy on a certain day, divided by the amount of capital allocated to the strategy.
$$
r_i = \frac{PnL_i}{Allocated \quad capital}
$$
So our expected return will be:
$$
\mu = \frac{1}{N}\sum\limits_{i=1}^N r_i
$$
On the other hand, a simple measure of risk will be the standard deviation of returns:
$$
\sigma = \frac{1}{N}\sum\limits_{i=1}^N (r_i - \mu)^2
$$

We will here make a fundamental assumption, since it is arguably extremely hard to forecast future returns of strategies, we will consider past returns as a proxy for future returns.\\