In this thesis we address the challenge of portfolio allocation applied to the context of algorithmic trading. The aim is to allocate risk to a subset of all the available algorithmic trading strategies based on mean-reversion on a weekly basis. The problem is to be solved in two separate parts: firstly find which strategies out of the many available to put into production at any moment in time and then assign proper risk weights to these strategies. We will see that both the steps are to be addressed with care as none of these problem if solved alone can achieve satisfactory results. All the results will be compared with a proper benchmark that mimics the current non-systematic allocation strategy. If the procedure will be implemented within the firm's trading framework it will make the whole investing process completely systematic. The weekly allocation period is chosen as it fits at best the characteristics of the market of interest and it avoids incurring in excessive transaction costs arousing from daily rebalancing of the portfolio that would erase any improvement given by the selection methodology.\\
Among the challenges that have been faced we should firstly mention the abundance of strategies that makes our search space highly dimensional and diverse. We should also remind of the well known issue that alpha in algorithmic trading strategies is not everlasting. First of all we define alpha as the additional return one gains over simply investing in the market. Going back to the alpha issue, it turns out that there is a point in time at which any strategy will stop working and will necessarily be switched off. On the other hand, as a reaction to market changes, some strategies that in the past performed poorly might become alpha generators. Achieving optimal timing in putting into production and switching off the strategies represents a challenge but also an opportunity to significantly increase trading performance. This task is hard to perform in other ways than algorithmic selection because often what is selected might not look intuitive to trade at first sight. To explain what we mean with "non-persistent alpha", that in other words means that inter-market relationships change through time, we make a simple example. For example, if one is trading on the well known relationship between Gold and Silver (which is expected to be steadily meaningful and potentially a good source of alpha), it might be that due to some specific event, or even a change in the microstructure of one of the two traded instruments this correlation breaks down, changing all the underlying market-dynamics and making the trading algorithm unprofitable. Moreover, in some cases, a certain strategy might perform really well for years until somebody in the market starts exploiting it systematically and at higher frequency than what hedge funds and small algorithmic trading firms can actually sustain. In such cases detecting a switching point in the performance of strategies is of crucial importance.\\