We went through the hard challenge of building a well-balanced portfolio from scratch on a set of strategies where most of them have bad performances. After the whole framework has been developed we realize that the most challenging and important part was the first one: selecting which strategies to put in production each week. We learnt that in this context robustness and simplicity are key to performance. Very complex models didn't seem to be ideal for our scope, on the other hand, methods that required little optimization worked very well. Moreover, it turned out that the ability of an algorithm to adapt to the changes in the market is pivotal to make the portfolio survive the shocks of the changes in the Market. As we have shown before, Brexit and the US Presidential Election caused deep changes in the relationship between futures, and many strategies based on mean reversion stopped working at that time. The method that proved to be the best, was the one that managed to survive in the out-of-sample tests to such changes, adapting his behaviour to the new dynamics.\\
Once the strategies have been switched the job remaining is fairly easy so we decided to increase the complexity of the models we tested. At this stage we preferred in the end a model that is more reliable and interpretable, with respect to a sophisticated but less controllable Genetic Portfolio Allocator. We managed indeed to improve the performance of a simple equally weighted portfolio and we controlled our performance better than with a traditional mean-variance portfolio. In the end, we have a couple of critical remarks to remind, first of all, the good performance we have shown in this project looks appealing, but we need to take into account the fact that if implemented in the "real world" these would suffer from the usual issues of slippage and technology problems (our results already include transaction fees). We estimated in fact, that in production roughly half of the PnL is eaten by slippage.\\
The second remark is more broad and covers the issue of innovation in algorithmic trading: as we outlined in the first part, the number of "good strategies" is decreasing with time as markets become more and more competitive. If on one hand we can accept a decrease of the performance of our models and of the number of strategies that actually work, the core lesson we need to take is that the underlying strategies must be continuously updated and improved to keep the pace with the competitiveness of financial markets. 

