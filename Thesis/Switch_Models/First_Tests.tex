We run different tests (in our in-sample period) to see which meaningful combination of features could come up with a proper switching model. At first, we tried to base our models only on one feature and it turned out that using only one feature was not enough as the data is really heterogeneous (i.e. the strategies are all different among each other in term of performance properties) and many strategies have very poor performance. In fact, there are many strategies that just make no sense from an economical point of view or just follow a trading logic that doesn\'t work when traded with transaction costs making their PnL line a constant negative line.
\todo{Una foto di una equity line negativissima}
This fact forced us to somehow use a second feature to act as a filter that would wipe out of the full set of strategies a huge majority that had not been performing well in the past. We directed our endeavours towards finding this meaningful filter of strategies. What this filter has to do, is to look at the past performance of any single strategy and set a threshold below which even if the current performance is good this strategy would not be switched on. The reason for this is that many strategies have some very short period where they work well due to specific market conditions that don't last for long. The filter would try to ensure the strategy had significantly long periods when the performance was good, as a proof of consistency. We luckily have a huge wealth of strategies and we can afford being strict in selecting strategies giving more strength to our method. After some tests and discussions we decided that a good filter is given by a long-window rolling Sharpe-ratio. This feature looks at the performance of a strategy and computes with a certain rolling window the Sharpe. Then we look at the resulting time series and we require the Sharpe-ratio to be sufficiently good over a certain window in history. In other words, every Monday the historical x-days Sharpe-ratio for each strategy is computed and if we are able to find  a period of x-days when a strategy performed sufficiently well in terms of Sharpe we believe this is a strategy that can be switched on and off in the future. Evidence will show that this lookback window should be quite long, proving that the strategy has been working for quite a long time in the past (See following parts). Once this filter is applied, the remaining strategies can be selected according to information coming from another feature that we will explore in the following part.\\
Armed with this tool we started building many different models with all the features we had at hand. Results showed that the Random Forest tree was almost always right in estimating the predictive power of features. In fact, we noticed that portfolios based on features like Exponential Moving Averages and Hit Ratio (features with low predictive power) didn't perform as well as portfolios based on Sharpe-ratio.

Moreover, an interesting aspect to analyse is the set of strategies switched by these methods, it seems that they tend to select very similar strategies as the ones switched by the Sharpe-Ratio based portfolios, but with much worse market timing.\\