The first method we try is purely based on the Sharpe Ratio, we simply look for strategies that satisfied a certain requirement in the past and eventually switch them on based on the current Sharpe ratio over a certain window. More in detail, every week we scan the whole universe of strategies and we apply the aforementioned sharpe-filter, pre-selecting only the ones that had a "good enough" performance over a certain window at any point in the past. That means this is a strategy that somehow has proven to be an alpha generator in the past at least. After this pre-selection we put in production only the strategies that currently have a sharpe over a certain window greater than a fixed threshold. For this method to work properly we need to fit four parameters (two thresholds and two windows for the Sharpe Ratio). Even though we might let the optimizer work on an infinite space, we set some boundaries to our search. For example, we will require the sharpe filter to work over a longer period (at least 6 months). While the shorter sharpe that ranks, will be based on a shorter window that will not exceed one trading year.\\
We run the in-sample grid search and nicely find a smooth surface. This is good news because it confirms that we are not chasing the noise but generating a real signal. 

%\begin{figure}[htbp]
%	\centering
%	\includegraphics[width=0.6\textwidth]{Figures/moving_average.eps}
%	\caption{Rolling average of the returns for 4 industries and Rf}
%	\label{rolling_average}
%\end{figure}

\todo{Al primo giro spiega un po' come interpretare il grafico}

The results are definitely interesting. Since we don't care about finding exactly the perfect optimal combination of parameters we stick with what we believe makes sense in the optimal area (the lighter one in Figure 1).