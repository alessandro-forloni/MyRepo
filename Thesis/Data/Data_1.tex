Here we aim at describing our data in terms of nature, size and structure.\\
To be precise, we have two datasets, one for in-sample analysis and one for out-of-sample analysis. The in-sample dataset consists of 13000 simulated strategies based on mean-reversion (13000 columns). All these strategies are trading one futures against another one looking for relative mispricings. This number of strategies comes out of a simulation of almost all possible trading pairs among roughly 150 futures traded worldwide. The huge diversity among these strategies makes it hard to find a unique model to allocate risk among all of them. Diversification has to be applied not only to asset classes, but also taking into account type of algorithms, trading latency and underlying country or region of exposure.\\
The data spans through almost six years, from January 2012 to August 2017. To perform the studies, the final year and a half has been dropped to be used as part of the final validation set (also referred to as production set).
For what concerns our out-of-sample data we have a dataset of 18000 strategies, still based on mean reversion ranging from 2012 to May 2018. This set is an extension of the previous one, generated once some futures were added to the list of the ones we are allowed to trade. The reason why we use this out-of-sample set is to have an out-of-sample both in terms of time (two years and a half untouched by optimization) and an out-of-sample in terms of strategies so that we somehow test how our models will work in an untouched territory.\\
We applied some cleaning for both the files and decided to remove any strategy involved in Swiss franc trading, to avoid our results to be biased by the famous drop that happened on the 15th of January 2015. We don't want to penalize or advantage any strategy that happened to be trading the Swiss Franc in either long or short side in that day because we believe that was a statistically unpredictable event. There is no guarantee that such an event could be forecasted only with information coming from strategy performance.\\
