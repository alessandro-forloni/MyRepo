The first method we try to implement is a Genetic-Leaning portfolio allocator. The method is based on the idea of making the algorithm evolve to find an optimal allocation through extensive genetic mutation. \todo{DESCRITTO DA QUALCUNO, CITA ATRICOLI} The approach is rather brute-force as it tries to test as many portfolios as possible until the optimal one is found. A more human-like analogy is the following: the algorithm acts as a boss letting many portfolio managers allocate risk according to their views. As time passes the boss will evaluate the portfolio managers based on specific performance measures (that are not only raw pnl) and kicks out the worst performing. At each stage he tries to replace the worst portfolio managers with completely new ones and with a set of managers that trade similarly to the best ons. Let's dig into the underlying methodology: on each monday we face the challenge of assigning weights (between 0 and 1) to the set of tradable strategies. The algorithm is initialized  with a set of random portfolios $\mathbf{w} = (\mathbf{w_1} \dots \mathbf{w_N})$, where each $\mathbf{w_1}$ represents a feasible allocation of risk (we will generate uniformly distributed random weights where 1 represents the maximum risk that can be allocated to a strategy and zero means that no risk is allocated to the strategy). The algorithm lets these portfolios trade over a certain window in the past and evaluates their performance. Once all of them have traded, the algorithm ranks them assigning a score given by a so-called \textit{Fitness Function}, which takes many metrics into account to evaluate a portfolio. Then the algorithm kicks out the worst performing, and substitutes them with a new generation (details on this part will be explaine later).\\ The procedure is repeated until an optimum is reached, or in other terms this optimizer is not able to find better portfolios. At this point the final portfolio will be an average of the best found portfolios.\\
The name Genetic comes from the idea that natural selection and evolution are applied to the set of portfolios. If a portfolio is just bad it will not survive the selection step, while if a portfolio is good it will be challenged with a muted version of itself that might represent an evolutional step. This kind of approach has pros and cons, let's first evaluate the positive aspects:
\begin{itemize}
	\item The optimization carried in a way that is able to be conducted in a multidimentional space, in different local minima in parallel  avoiding the risk of missing a global minimum. The mutation happens in a way that optimization is more refined in well performing areas, while it is also randomized to cover the whole space
	\item The algorithm is conceived in such a way that it serves really well our needs and requirements. Evaluating portfolios with the so called Fitness Function it allows to penalize portfolios that perform well but that give rise to the typical issues of portfolio optimization like instability of weights, poor diversification or meaningless negative weights. The optimization is already done without having to worry about any type of complex mathematical formulation to impose constraints.
	\item The approach requires very little parameters: the length of the lookback window and the weights to give to any performance measure used to assess portfolio performance.
	\item The algorithm might fully embrace the non-linearity of the problem and autonomously find relationships between strategies that other methods might not find. 
\end{itemize}  

On the other hand this method has some drawbacks:

\begin{itemize}
	\item This brute-force algorithm requires an enormous computing power to span the whole space and rank all the portfolio. Needless to say, we will notice later that the more computing time is given to the algoithm the more the randomness in it is limited and the performance improves. We will dig later in this aspect. \todo{CREA GRAFICO DI PERFORMANCE AL VARIARE DI N}
	\item As outlined above, there is some randomness, as most of the portfolios that are tested are just randomly generated, so there is little chance to find a precise optimum, but rather something that is quite close to it
	\item The algorithm looks backward and makes the assumption that the best performing combination in the past will still be the best for the next week.
	\item Even though there are few parameters to be set, the algorithm is quite sensitive to these values.
\end{itemize}