Let's now address the issue of defining the fitness function. We will indicate this function with the symbol $f_f$, it is a simple $\mathbb{R}^N \rightarrow \mathbb{R}$ that given a portfolio vector $\mathbf{w}$ returns a real number as a score. This function is the core of the whole algorithm, because it can evaluate a portfolio based on its performance but also based on how the portfolio fits our requirements. This core function will look for a particular optimum that is charachterized by a portfolio vector that scores well according to different metrics. For example it might penalize a portfolio that assigns a lot of weight to few strategies, or a portfolio that changes too much compared to what was traded the previous week. Defining this function in the proper way takes more intuition than calculation, and requires to pay attention to a couple of details. Of course, the more complex the fitness function the more our taste can be satisfied, but also the more computational time is required.\\
To make out optimization process faster and more realistic, we will restrict the possible risk allocation values to a finite set. We will give an allocation score that ranges from 0 to 1 in steps of 0.1. Doing so we reduce the infinite set of possible continuous allocations and we make the optimization faster and more reliable. Moreover, we aim to simulate the behaviour of a human portfolio manager that allocates manually, in such a case one would rationally think of allocating risk in a stepwise fashion with rounded numbers.\\ 

\todo{Qui metti tipo di randomizzazione/replacement}\\
\todo{Metti dettagli per definire l'ottimo}\\