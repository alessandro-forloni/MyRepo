Let's now address the issue of defining the fitness function. We will indicate this function with the symbol $f_f$, it is a simple $\mathbb{R}^N \rightarrow \mathbb{R}$ that given a portfolio vector $\mathbf{w}$ returns a real number as a score. This function is the core of the whole algorithm, because it can evaluate a portfolio based on its performance but also based on how the portfolio fits our requirements. This core function will look for a particular optimum that is charachterized by a portfolio vector that scores well according to different metrics. For example it might penalize a portfolio that assigns a lot of weight to few strategies, or a portfolio that changes too much compared to what was traded the previous week. Defining this function in the proper way takes more intuition than calculation, and requires to pay attention to a couple of details. Of course, the more complex the fitness function the more our taste can be satisfied, but also the more computational time is required.\\

\todo{Qui metti tipo di randomizzazione/replacement}\\
\todo{Metti dettagli per definire l'ottimo}\\