Here we propose an alternative model, we look for something more controllable, interpretable and less computationally heavy. The idea is to oppose the genetic learner with something deeply different.\\
If we look at the drawbacks of Genetic Portfolios described in section 4.1, we will be able to overcome them in this section. In fact, the computational power required is much lower, the solution is exact in the sense that there is no randomness, but moreover there are very few parameters in the algorithm, and the dependency on these parameters is not that relevant.\\
Let's dig a bit in how this portfolio is going to be built. As an alternative to the genetic portfolio, we will start from the same point, that is a given allocation of strategies for a given week. We will try to achieve a minimum variance approach, of course for all the reasons outlined above we will not try to use directly a Markowitz approach (even though this will be considered as a benchmark). Given an allocation we will try to minimize the risk by looking for similar strategies and penalizing their weight in production to avoid having a portfolio that has an excessive exposure to a set of similar strategies (i.e. belonging to the same sector/region/asset-class). This should ideally give a well-diversified portfolio. We will at last look at the sharpe ratio to tilt the weight towards better performing strategies.\\
More in detail our approach will be to cluster the strategies, form groups within similar strategies and then allocate an equal weight for each group. Then weights within each group are tilted based on recent performance of strategies.

