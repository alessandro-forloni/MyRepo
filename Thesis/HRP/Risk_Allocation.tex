Once the strategies in production for the week have been preprocessed and clustered, we assign to them weights using the information coming from clusters. This means that we will assign the same risk weight to each cluster (i.e one over the number of clusters for each cluster). This is like a generalized risk-parity, where the parity is based on "similarity" among strategies. If we have many strategies belonging to one cluster (that means they are quite similar in terms of performance) we don't want to have an excessive weight on all of them because this would expose too much our portfolio to one risk factor. on the other hand we might like to give more weight to one strategy that behaves really differently from the rest of the cohort improving diversification and reducing drops.\\
Once weights to each cluster are given, the weights within the clusters must be assigned. Here we have an infinite array of possibilities, but we decided to stick with something really simple: assigning weights within the cluster based on the Sharpe Ratio. This means that the risk allocated to the cluster is splitted among the strategies in the cluster in a way that it reflects the recent performance of the strategies. If a strategy in the cluster has had a very bad sharpe ratio in te last month maybe it will have a weight of zero in production. On the contrary, a strategy that is working really well might get almost all the risk of the cluster on itself. Given a vector of weights for the strategies in each cluster, the resulting formula will be: 

\begin{equation}\label{HRP_weights}
\mathbf{w_i} =  \frac{1}{\# Clusters}\left(\frac{Sharpe_i}{\sum_j |Sharpe_j|}\right)
\end{equation}