We want to check that the results coming from the Random Forest tree are actually meaningful. To this end, we will observe the problem from a slightly different angle. We want to verify whether the feature that is supposedly the best will actually have some predictive power for our portfolio.\\
To this end we will run a simple regression where we will try to predict the future one-week return of a strategy given one feature (the best one that is a rolling sharpe over a window of 350 trading days), in our case we will try to fit:

\begin{equation} \label{regression}
r_i = \beta SR\_350_i + \epsilon_i \qquad i=1\dots N
\end{equation}

Where N is the number of traded strategies. The regression is run on the whole in-sample period. We will evaluate the performance of the feature by looking at the $R^2$ score and the individual t-statistic.

\todo{QUI IMMAGINOZZA DEL TEST}

Remember that to accept the the alternative hypothesis of the feature being relevant at a 95\% level we need to have the t-statistic to be greater than 1.65. 