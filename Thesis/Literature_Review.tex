The problem of portfolio optimization is one of the oldest and most discussed topics in Finance. The traditional theory that has been considered the milestone of Finance for decades has been developed at first by Harry Markowitz who won the Nobel Prize for his article \textit{Portfolio Selection} in 1952 \cite{Markowitz}. The Modern Portfolio Theory that was developed in those years is actually still regarded as the baseline for portfolio managers. The idea is to look at portfolio allocation in a return-risk trade-off context where risk is estimated by past volatility of returns. Unfortunately, Markowitz allocation procedures have some drawbacks, mainly relative to the numerical instability of the estimation of asset returns, their volatility and their correlations. Some scholars improved through the years the models trying to add stability. In 1992 a huge step forward was made by Fisher Black and Robert Litterman with their article \textit{Global Portfolio Optimization} \cite{black_litterman} that merged the CAPM equilibrium theory with the mean-variance optimization proposed by Markowitz. Their idea was that incorporating meaningful information coming from CAPM and personal views of portfolio managers would solve the issues of quantitative portfolios. More recently, an additional step was made by Olivier Ledoit and Michael Wolf whose contribution was to improve the global performance of covariance matrices by introducing the idea of \textit{shrinkage}. This method allows to have more stable covariance matrices and therefore making the Markowitz framework more stable and applicable.  \\
Many experts from different fields have worked to enrich the knowledge in this specific area. A well-known example is that of Professor Cover, whose work is recognized as one of the finest attempts to use signal theory in portfolio allocation. With his article \textit{Universal Portfolios} \cite{universal_portfolios} he builds a portfolio, that in terms of performance, asympthotically beats the best stock over a given set of stocks. The interesting part of cover's work is that he attempts to solve the portfolio optimization puzzle in a non-parametric way, using robust results, this unfortunately comes at the expense of not universal applicability.\\
Recently, with the advent of Machine Learning, many experts started applying powerful algorithms to portfolio selection with interesting results. Any kind of use has been made, from forecasting returns to allocate risk to cluster assets creating well-diversified portfolios. The increase of computational power has allowed to test on large-scale portfolios complex algorithms like genetic learning models or neural networks. These have been used in many ways, for example some researchers in the US have trained a one-layer recurrent neural network to dynamically optimize a portfolio \cite{NN_1}. Others tried to forecast asset returns with a specific Hopfield Neural network to input into a traditional mean-variance Markowitz style optimization \cite{NN_2}.\\
Despite their out-of-sample results are not outstanding, these pieces of work set with others a new path for portfolio optimization that extracts the most information out of the available data. We will follow this path trying to optimize our portfolio using the most information as possible. In particular we will follow the approach of some researches in the area of genetic algorithms \cite{genetic_1} and that of Marcos Lopez de Prado, that aims at building well-diversified portfolios with unsupervised clustering methods \cite{HRP}.\\
More than 50 years after the first attempt to address the issue of portfolio selection, Markowitz models are still regarded as the baseline model around which all the theory is built. These models still have relevant real-world issues such as ignorance of transaction costs and high instability of the portfolio, but are really intuitive and representative of the dynamics of a rational investor.  